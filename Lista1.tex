% Options for packages loaded elsewhere
\PassOptionsToPackage{unicode}{hyperref}
\PassOptionsToPackage{hyphens}{url}
%
\documentclass[
]{article}
\usepackage{amsmath,amssymb}
\usepackage{iftex}
\ifPDFTeX
  \usepackage[T1]{fontenc}
  \usepackage[utf8]{inputenc}
  \usepackage{textcomp} % provide euro and other symbols
\else % if luatex or xetex
  \usepackage{unicode-math} % this also loads fontspec
  \defaultfontfeatures{Scale=MatchLowercase}
  \defaultfontfeatures[\rmfamily]{Ligatures=TeX,Scale=1}
\fi
\usepackage{lmodern}
\ifPDFTeX\else
  % xetex/luatex font selection
\fi
% Use upquote if available, for straight quotes in verbatim environments
\IfFileExists{upquote.sty}{\usepackage{upquote}}{}
\IfFileExists{microtype.sty}{% use microtype if available
  \usepackage[]{microtype}
  \UseMicrotypeSet[protrusion]{basicmath} % disable protrusion for tt fonts
}{}
\makeatletter
\@ifundefined{KOMAClassName}{% if non-KOMA class
  \IfFileExists{parskip.sty}{%
    \usepackage{parskip}
  }{% else
    \setlength{\parindent}{0pt}
    \setlength{\parskip}{6pt plus 2pt minus 1pt}}
}{% if KOMA class
  \KOMAoptions{parskip=half}}
\makeatother
\usepackage{xcolor}
\usepackage[margin=1in]{geometry}
\usepackage{color}
\usepackage{fancyvrb}
\newcommand{\VerbBar}{|}
\newcommand{\VERB}{\Verb[commandchars=\\\{\}]}
\DefineVerbatimEnvironment{Highlighting}{Verbatim}{commandchars=\\\{\}}
% Add ',fontsize=\small' for more characters per line
\usepackage{framed}
\definecolor{shadecolor}{RGB}{248,248,248}
\newenvironment{Shaded}{\begin{snugshade}}{\end{snugshade}}
\newcommand{\AlertTok}[1]{\textcolor[rgb]{0.94,0.16,0.16}{#1}}
\newcommand{\AnnotationTok}[1]{\textcolor[rgb]{0.56,0.35,0.01}{\textbf{\textit{#1}}}}
\newcommand{\AttributeTok}[1]{\textcolor[rgb]{0.13,0.29,0.53}{#1}}
\newcommand{\BaseNTok}[1]{\textcolor[rgb]{0.00,0.00,0.81}{#1}}
\newcommand{\BuiltInTok}[1]{#1}
\newcommand{\CharTok}[1]{\textcolor[rgb]{0.31,0.60,0.02}{#1}}
\newcommand{\CommentTok}[1]{\textcolor[rgb]{0.56,0.35,0.01}{\textit{#1}}}
\newcommand{\CommentVarTok}[1]{\textcolor[rgb]{0.56,0.35,0.01}{\textbf{\textit{#1}}}}
\newcommand{\ConstantTok}[1]{\textcolor[rgb]{0.56,0.35,0.01}{#1}}
\newcommand{\ControlFlowTok}[1]{\textcolor[rgb]{0.13,0.29,0.53}{\textbf{#1}}}
\newcommand{\DataTypeTok}[1]{\textcolor[rgb]{0.13,0.29,0.53}{#1}}
\newcommand{\DecValTok}[1]{\textcolor[rgb]{0.00,0.00,0.81}{#1}}
\newcommand{\DocumentationTok}[1]{\textcolor[rgb]{0.56,0.35,0.01}{\textbf{\textit{#1}}}}
\newcommand{\ErrorTok}[1]{\textcolor[rgb]{0.64,0.00,0.00}{\textbf{#1}}}
\newcommand{\ExtensionTok}[1]{#1}
\newcommand{\FloatTok}[1]{\textcolor[rgb]{0.00,0.00,0.81}{#1}}
\newcommand{\FunctionTok}[1]{\textcolor[rgb]{0.13,0.29,0.53}{\textbf{#1}}}
\newcommand{\ImportTok}[1]{#1}
\newcommand{\InformationTok}[1]{\textcolor[rgb]{0.56,0.35,0.01}{\textbf{\textit{#1}}}}
\newcommand{\KeywordTok}[1]{\textcolor[rgb]{0.13,0.29,0.53}{\textbf{#1}}}
\newcommand{\NormalTok}[1]{#1}
\newcommand{\OperatorTok}[1]{\textcolor[rgb]{0.81,0.36,0.00}{\textbf{#1}}}
\newcommand{\OtherTok}[1]{\textcolor[rgb]{0.56,0.35,0.01}{#1}}
\newcommand{\PreprocessorTok}[1]{\textcolor[rgb]{0.56,0.35,0.01}{\textit{#1}}}
\newcommand{\RegionMarkerTok}[1]{#1}
\newcommand{\SpecialCharTok}[1]{\textcolor[rgb]{0.81,0.36,0.00}{\textbf{#1}}}
\newcommand{\SpecialStringTok}[1]{\textcolor[rgb]{0.31,0.60,0.02}{#1}}
\newcommand{\StringTok}[1]{\textcolor[rgb]{0.31,0.60,0.02}{#1}}
\newcommand{\VariableTok}[1]{\textcolor[rgb]{0.00,0.00,0.00}{#1}}
\newcommand{\VerbatimStringTok}[1]{\textcolor[rgb]{0.31,0.60,0.02}{#1}}
\newcommand{\WarningTok}[1]{\textcolor[rgb]{0.56,0.35,0.01}{\textbf{\textit{#1}}}}
\usepackage{graphicx}
\makeatletter
\def\maxwidth{\ifdim\Gin@nat@width>\linewidth\linewidth\else\Gin@nat@width\fi}
\def\maxheight{\ifdim\Gin@nat@height>\textheight\textheight\else\Gin@nat@height\fi}
\makeatother
% Scale images if necessary, so that they will not overflow the page
% margins by default, and it is still possible to overwrite the defaults
% using explicit options in \includegraphics[width, height, ...]{}
\setkeys{Gin}{width=\maxwidth,height=\maxheight,keepaspectratio}
% Set default figure placement to htbp
\makeatletter
\def\fps@figure{htbp}
\makeatother
\setlength{\emergencystretch}{3em} % prevent overfull lines
\providecommand{\tightlist}{%
  \setlength{\itemsep}{0pt}\setlength{\parskip}{0pt}}
\setcounter{secnumdepth}{-\maxdimen} % remove section numbering
\usepackage{booktabs}
\usepackage{longtable}
\usepackage{array}
\usepackage{multirow}
\usepackage{wrapfig}
\usepackage{float}
\usepackage{colortbl}
\usepackage{pdflscape}
\usepackage{tabu}
\usepackage{threeparttable}
\usepackage{threeparttablex}
\usepackage[normalem]{ulem}
\usepackage{makecell}
\usepackage{xcolor}
\ifLuaTeX
  \usepackage{selnolig}  % disable illegal ligatures
\fi
\IfFileExists{bookmark.sty}{\usepackage{bookmark}}{\usepackage{hyperref}}
\IfFileExists{xurl.sty}{\usepackage{xurl}}{} % add URL line breaks if available
\urlstyle{same}
\hypersetup{
  pdftitle={Lista de Exercícios 1},
  pdfauthor={João Victor de Oliveira Nogueira},
  hidelinks,
  pdfcreator={LaTeX via pandoc}}

\title{Lista de Exercícios 1}
\usepackage{etoolbox}
\makeatletter
\providecommand{\subtitle}[1]{% add subtitle to \maketitle
  \apptocmd{\@title}{\par {\large #1 \par}}{}{}
}
\makeatother
\subtitle{Introdução - Aprendizado Supervisionado e Não Supervisionado}
\author{João Victor de Oliveira Nogueira}
\date{15-04-2024}

\begin{document}
\maketitle

\subsubsection{1. (a) Análise Estatística
Paramétrica}\label{a-anuxe1lise-estatuxedstica-paramuxe9trica}

O artigo tem como objetivo ajustar modelos de predição lineares para
estimar o número esperado de gols para times mandantes e visitantes,
estimando pârametros de força de ataque e de defesa.

O artigo traz o ajuste de um modelo de Poisson para estimar os
parâmetros de força de ataque e defesa dos times mandantes e visitantes.
O modelo de Poisson é um modelo paramétrico e por isso o link com a
pergunta da lista.

Durante o artigo é mostrado diversas tabelas com as estimativas e também
é mostrado algumas probabilidades de vitória, empate e derrota para os
times mandantes e visitantes além de probabilidades de resultados
exatos.

Achei muito interrensante essa abordagem por ser feita com algo que eu
gosto bastante além de que modelos de previsão de eventos é algo que
pode ser usado de diversas maneiras no nosso ramo. O artigo também cita
as diversas maneiras que se pode estimar os resultados, além de deixar
claro que teriam modelos melhores e que foi usado um modelo simples que
performou muito bem.

Também gostei da forma com que os dados foram analisados, mostrou como é
importantes também conhescer sobre o assunto que está sendo tratado, no
caso futebol deu para notar que os parâmetros defensivos de um time
tendem ser mais importante do que os ofensivos quando se observa a
colocação dos times no campeonato.

Fonte:
\textbf{\href{https://biometria.ufla.br/index.php/BBJ/article/view/403/253}{ESTIMATING
RATINGS IN FOOTBALL: BRAZILIAN CHAMPIONSHIP 2017}}

\subsubsection{(c) Reconhecimento de Padrões e (d) Aprendizado de
Máquinas e/ou Estatístico
Supervisionado}\label{c-reconhecimento-de-padruxf5es-e-d-aprendizado-de-muxe1quinas-eou-estatuxedstico-supervisionado}

O artigo tem como objetivo aplicar métodos de aprendizado de máquina
para classificar e analisar o desfecho de pacientes com COVID-19 que
receberam alta ou óbito e descrever o perfil dos pacientes infectados
pelo vírus.

Os dados utilizados tem variáreis de entrada, sendo: sexo, idade, tempo
de internação, tipo de atendimento (ambulatorial, pronto atendimento,
externo e interno) e unidade federativa a que pertence o paciente, e
como desfecho a variável de saída que é a alta ou óbito do paciente.

O artigo faz a comparação de 4 algoritmos de classificação: Máquina de
vetores de suporte (SVM), K-vizinhos mais próximos (KNN), Redes Naïve
Bayes e Random florest. Todos esses algoristimos são não paramétricos e
supervisionados e essas caracteristicas são as que linkam com a pergunta
da lista.

Achei interessante a abordagem do artigo em comparar os algoritmos de
classificação, os dados utilizados tinham uma grande diferença entre as
duas possíveis saídas, o número de óbitos era de apenas 23 enquanto o de
altas era de 3879. Isso pode ser um problema comum em futuras análises
que um estatístico possa fazer, então a abordagem de comparar os modelos
demonstra a importância de se escolher um modelo que se adeque melhor
aos dados.

No final do artigo é feita uma análise dos resultados dos modelos e o
modelo de Redes Naïve Bayes apresentaram um desempenho superior e
capacidade de calcular indicadores de acurácia, como sensibilidade,
especificidade e coeficiente kappa, os outros modelos não conseguiram
acertar a classificação de nenhum óbito. Esses resultados são úteis caso
em alguma ocasião futura eu tenha um caso parecido de classificação
binária com um desbalanceamento grande entre as classes.

Fonte:
\textbf{\href{https://biometria.ufla.br/index.php/BBJ/article/view/588/362}{Classification
and Analysis of Patients with COVID-19 Using Machine Learning}}

\subsubsection{2. Considere um hipercubo de dimensão r e lados de
comprimento 2A. Dentro deste hipercubo temos uma hiperesfera
r-dimensional de raio A. Encontre a proporção do volume do hipercubo que
está fora da hiperesfera e mostre que a proporção tende a 1 a medida que
a dimensão r cresce. Escreva um programa R para verificar o resultado
encontrado. O que este resultado
significa?}\label{considere-um-hipercubo-de-dimensuxe3o-r-e-lados-de-comprimento-2a.-dentro-deste-hipercubo-temos-uma-hiperesfera-r-dimensional-de-raio-a.-encontre-a-proporuxe7uxe3o-do-volume-do-hipercubo-que-estuxe1-fora-da-hiperesfera-e-mostre-que-a-proporuxe7uxe3o-tende-a-1-a-medida-que-a-dimensuxe3o-r-cresce.-escreva-um-programa-r-para-verificar-o-resultado-encontrado.-o-que-este-resultado-significa}

Primeiramente temos que o volume do hipercubo é dado por
\(V_{cubo} = (2A)^r\) e o volume da hiperesfera é dado por
\(V_{esfera} = \frac{\pi^{r/2}}{\Gamma(r/2 + 1)}A^r\). A proporção do
volume do hipercubo que está fora da hiperesfera é dada por
\(P = 1 - \frac{V_{esfera}}{V_{cubo}}\). Irei calcular a proporção para
r = 1, 2, \ldots, 40 e verificar se a proporção tende a 1 a medida que a
dimensão r cresce.

O volume do hipercubo segue \(V_{cubo} = (2A)^r\), já a hiperesfera
segue \(V_{esfera} = \frac{\pi^{r/2}}{\Gamma(r/2 + 1)}A^r\). Para
encontrar a proporação do volume do hipercubo que está fora da
hipersfera é necessário seguir \(P = 1 - \frac{V_{esfera}}{V_{cubo}}\).

\begin{Shaded}
\begin{Highlighting}[]
\FunctionTok{options}\NormalTok{(}\AttributeTok{scipen =} \DecValTok{999999}\NormalTok{)}

\NormalTok{r }\OtherTok{=} \DecValTok{1}\SpecialCharTok{:}\DecValTok{40}
\NormalTok{A }\OtherTok{=} \DecValTok{1}
\NormalTok{vCubo }\OtherTok{=}\NormalTok{ (}\DecValTok{2}\SpecialCharTok{*}\NormalTok{A)}\SpecialCharTok{\^{}}\NormalTok{r}
\NormalTok{vEsfera }\OtherTok{=}\NormalTok{ (pi}\SpecialCharTok{\^{}}\NormalTok{(r}\SpecialCharTok{/}\DecValTok{2}\NormalTok{))}\SpecialCharTok{/}\NormalTok{(}\FunctionTok{gamma}\NormalTok{(r}\SpecialCharTok{/}\DecValTok{2} \SpecialCharTok{+} \DecValTok{1}\NormalTok{))}\SpecialCharTok{*}\NormalTok{A}\SpecialCharTok{\^{}}\NormalTok{r}
\NormalTok{P }\OtherTok{=} \DecValTok{1}\SpecialCharTok{{-}}\NormalTok{vEsfera}\SpecialCharTok{/}\NormalTok{vCubo}

\FunctionTok{library}\NormalTok{(ggplot2)}

\NormalTok{data }\OtherTok{=} \FunctionTok{data.frame}\NormalTok{(}\AttributeTok{n =}\NormalTok{ r, }\AttributeTok{P =}\NormalTok{ P)}
\FunctionTok{ggplot}\NormalTok{(data, }\FunctionTok{aes}\NormalTok{(}\AttributeTok{x =}\NormalTok{ r, }\AttributeTok{y =}\NormalTok{ P)) }\SpecialCharTok{+}
  \FunctionTok{geom\_line}\NormalTok{() }\SpecialCharTok{+}
  \FunctionTok{geom\_line}\NormalTok{(}\AttributeTok{data =} \FunctionTok{data.frame}\NormalTok{(}\AttributeTok{x =} \FunctionTok{c}\NormalTok{(}\DecValTok{34}\NormalTok{, }\DecValTok{34}\NormalTok{),}\AttributeTok{y =} \FunctionTok{c}\NormalTok{(}\DecValTok{0}\NormalTok{, }\DecValTok{1}\NormalTok{)), }\FunctionTok{aes}\NormalTok{(}\AttributeTok{x =}\NormalTok{ x, }\AttributeTok{y =}\NormalTok{ y), }
            \AttributeTok{color =} \StringTok{"red"}\NormalTok{) }\SpecialCharTok{+}
  \FunctionTok{geom\_text}\NormalTok{(}\FunctionTok{aes}\NormalTok{(}\AttributeTok{x =} \DecValTok{34}\NormalTok{, }\AttributeTok{y =} \FloatTok{0.5}\NormalTok{,}\AttributeTok{label =} \StringTok{"Proporção = 1 }\SpecialCharTok{\textbackslash{}n}\StringTok{ r = 34            "}\NormalTok{), }
            \AttributeTok{color =} \StringTok{"red"}\NormalTok{, }\AttributeTok{hjust =} \DecValTok{1}\NormalTok{) }\SpecialCharTok{+}
  \FunctionTok{scale\_x\_continuous}\NormalTok{(}\AttributeTok{breaks =} \FunctionTok{seq}\NormalTok{(}\DecValTok{0}\NormalTok{, }\DecValTok{40}\NormalTok{, }\DecValTok{5}\NormalTok{)) }\SpecialCharTok{+}
  \FunctionTok{labs}\NormalTok{(}\AttributeTok{x =} \StringTok{"Dimensão r"}\NormalTok{,}
       \AttributeTok{y =} \StringTok{"Proporção"}\NormalTok{) }\SpecialCharTok{+} 
  \FunctionTok{theme\_minimal}\NormalTok{()}
\end{Highlighting}
\end{Shaded}

\includegraphics{Lista1_files/figure-latex/unnamed-chunk-1-1.pdf}

A proporção do volume do hipercubo que está fora da hiperesfera tende a
1 a medida que a dimensão r cresce, com um r = 10 o valor já fica muito
proximo de 1 mas o valor 1 é realmente atingido quando r = 34. Isso
significa que a hiperesfera se torna cada vez mais insignificante em
relação ao hipercubo.

Isso ocorre porque ao comparar as formulas do volume do hipercubo e da
hiperesfera, percebemos que os volumes tem em comum o termo \(A^r\), mas
no volume do hipercubo o termo é multiplicado por \(2^r\) enquanto no
volume da hiperesfera o termo é multiplicado por
\(\frac{\pi^{r/2}}{\Gamma(r/2 + 1)}\). Fiz uma tabela para compararmos a
diferença entre \(2^r\) e \(\frac{\pi^{r/2}}{\Gamma(r/2 + 1)}\) conforme
o valor de r cresce.

\begin{table}[!h]
\centering
\begin{tabular}{rrrr}
\toprule
r & \makecell[c]{Cubo \\ $2^r$} & \makecell[c]{Esfera \\ $\frac{\pi^{r/2}}{\Gamma(r/2 + 1)}$} & \makecell[c]{Termo \\ $A^2$ \\ $A=1$}\\
\midrule
1 & 2 & 2.0000000 & 1\\
2 & 4 & 3.1415927 & 1\\
3 & 8 & 4.1887902 & 1\\
4 & 16 & 4.9348022 & 1\\
5 & 32 & 5.2637890 & 1\\
\addlinespace
6 & 64 & 5.1677128 & 1\\
7 & 128 & 4.7247660 & 1\\
8 & 256 & 4.0587121 & 1\\
9 & 512 & 3.2985089 & 1\\
10 & 1024 & 2.5501640 & 1\\
\addlinespace
20 & 1048576 & 0.0258069 & 1\\
30 & 1073741824 & 0.0000219 & 1\\
40 & 1099511627776 & 0.0000000 & 1\\
\bottomrule
\end{tabular}
\end{table}

Com isso podemos concluir que a razão do volume do hipercubo que está
fora da hiperesfera tende a 1 a medida que a dimensão r cresce, pois o
volume do hipercubo cresce exponencialmente com a dimensão r, enquanto o
volume da hiperesfera decresce.

\textbf{Fonte:
\href{https://en.wikipedia.org/wiki/Volume_of_an_n-ball}{Volume
hiperesfera};
\href{https://www.quora.com/What-is-the-volume-of-a-four-dimensional-hypercube\#:~:text=For\%20a\%20hypercube\%20in\%20n,twice\%20the\%20number\%20of\%20dimensions.}{Volume
hipercubo}}

\end{document}
